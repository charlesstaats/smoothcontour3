% Copyright 2015 Charles Staats III
% 
% Licensed under the Apache License, Version 2.0 (the "License");
% you may not use this file except in compliance with the License.
% You may obtain a copy of the License at
% 
%     http://www.apache.org/licenses/LICENSE-2.0
% 
% Unless required by applicable law or agreed to in writing, software
% distributed under the License is distributed on an "AS IS" BASIS,
% WITHOUT WARRANTIES OR CONDITIONS OF ANY KIND, either express or implied.
% See the License for the specific language governing permissions and
% limitations under the License.

\documentclass{article}
\usepackage[T1]{fontenc}
\usepackage{lmodern}
\usepackage{asypictureB}
\usepackage{xcolor}
\usepackage{needspace}
\usepackage{listings}
\usepackage[justification=raggedright]{caption}
\usepackage{subcaption}
\usepackage{tikz}
\usetikzlibrary{spy}

%\setlength{\emergencystretch}{1em}
\tolerance=1000

% The maximum fraction of floats allowed at the top of a page:
\renewcommand{\topfraction}{0.9}

\lstset{language=C,
	breaklines=true,
	basicstyle=\ttfamily,
	breakatwhitespace=true,
	columns=flexible,
	keepspaces=true,
	numberblanklines=false,
	showstringspaces=false,
	commentstyle=\color{gray},
	backgroundcolor=\color{blue!10},
	frame=single,
	framerule=0pt,
	xleftmargin=3.01pt,
	xrightmargin=3.01pt,
	numberstyle=\footnotesize,
	showlines
	}
%
\newcommand{\mywidth}{}	
\newif\ifinminipage
%
\newcommand{\begincodelisting}{%
\end{minipage}%
\inminipagetrue%
\hfill
\begin{minipage}[t]{\dimexpr\linewidth-\mywidth-7pt\relax}
\strut\par\vspace*{-\baselineskip}
\lstset{aboveskip=0pt}
}
%
\newcommand{\breakcodelisting}{%
\end{minipage}%
\inminipagefalse%
\begingroup%
\lstset{aboveskip=0pt}
}
%
\newenvironment*{asyexample}[1]%
{\par\bigskip%
\renewcommand{\mywidth}{#1}
\noindent
\begin{minipage}[t]{\mywidth}%
\mbox{}\\[-\baselineskip]}%
{\ifinminipage\end{minipage}\else\endgroup\fi\par\medskip}

\usepackage[hidelinks]{hyperref}

\title{The \texttt{smoothcontour3} module for Asymptote}
\author{Charles Staats III}
\date{\today}

\begin{document}
\maketitle
\begin{abstract}
The \lstinline!smoothcontour3! module uses an adaptive algorithm to draw implicitly defined surfaces 
with smooth appearance. This documentation explains its use.
\end{abstract}
\tableofcontents
%
\section{Usage}
%
The API for this module consists of the single function \lstinline!implicitsurface()!:
\begin{lstlisting}
surface implicitsurface(real f(triple) = null, 
                        real ff(real,real,real) = null,
                        triple a, 
                        triple b,
                        int n = nmesh,
                        bool keyword overlapedges = false,
                        int keyword nx=n,
                        int keyword ny=n,
                        int keyword nz=n,
                        int keyword maxdepth = 8);
\end{lstlisting}
The function has three required\footnotemark{} 
parameters: a function $f$ (of type either 
\lstinline!real(triple)! or \lstinline!real(real,real,real)!) and two triples 
\lstinline!a! and !b!. It returns a \lstinline!surface! that 
approximates the zero locus of $f$ within the rectangular solid 
with opposite corners \lstinline!a! and \lstinline!b!.
%
\footnotetext{The syntax seems to say that 
\lstinline!f! and \lstinline!ff! are both optional parameters. 
However, the function will throw a runtime error unless exactly one of 
\lstinline!f!, \lstinline!ff! is set. This somewhat peculiar arrangement
was chosen so that the module works in all three of the following scenarios:
\begin{itemize}
\item The user wants to graph the zero set of a function
\lstinline!f(real,real,real)!.
\item The user wants to graph the zero set of a function
\lstinline!f(triple)!.
\item The user has overloaded a function name \lstinline!f!,
defining both \lstinline!f(triple)! and \lstinline!f(real,real,real)!.
\end{itemize}
While the first two cases could be covered by overloading 
\lstinline!implicitsurface!, that would cause an ambiguous call in the third 
case.}

\needspace{3\baselineskip}
Here are the explanations of the optional parameters:
\begin{itemize}
\item \lstinline!int n!---the number of initial segments in each of the 
$x$, $y$, $z$ directions. Defaults to \lstinline!nmesh!, which is usually 10.
\item \lstinline!bool overlapedges!---if \lstinline!true!, the patches of the surface are slightly enlarged to compensate for an artifact in which the viewer can see through the boundary between patches. (Some of this may actually be a result of edges not lining up perfectly, but I'm fairly sure a lot of it arises purely as a rendering artifact.) Defaults to \lstinline!false!. Keyword required---this parameter
can only be used in key-value format.
\item \lstinline!int nx!---overrides \lstinline!n! in the $x$ direction. 
Keyword required.
\item \lstinline!int ny!---overrides \lstinline!n! in the $y$ direction. 
Keyword required.
\item \lstinline!int nz!---overrides \lstinline!n! in the $z$ direction. 
Keyword required.
\item \lstinline!int maxdepth!---the maximum depth to which the adaptive algorithm will subdivide in an effort to find patches that closely approximate the true surface.  Keyword required.
\end{itemize}

\section{When should I use this module?}
The algorithm is designed to deal with a differentiable function with a smooth zero 
locus. Simple point singularities in the zero locus are usually not a problem,
but complex singularities (e.g. two tangent spheres) or one-dimensional 
singularities will take a long time to compute with no guarantee of a 
good result.
 
On one occasion I have seen it cope with a continuous, piecewise differentiable function, although it took a while to compute. Whatever you do, \emph{do not} 
pass it a piecewise constant function---it will take forever and return nothing good.

\section{Examples}
\subsection{Genus three surface}
\begin{figure}
\centering
\begin{minipage}[b]{\dimexpr\textwidth-9cm}
\subcaption{\lstinline!overlapedges=true!}
\end{minipage}
\begin{minipage}{8.8cm}%
% A tikzpicture is used here for spacing consistency with the next
% subfigure, which uses the tikz spy library.
\begin{tikzpicture}
\draw (0,0) node [above right] {%
\begin{asypicture}{name=genus_three}
settings.outformat = "png";
settings.render = 4;
size(8cm);

import smoothcontour3;


// Erdos lemniscate of order n:
real erdos(pair z, int n) { return abs(z^n-1)^2 - 1; }

real h = 0.12;

// Erdos lemniscate of order 3:
real lemn3(real x, real y) { return erdos((x,y), 3); }

// "Inflate" the order-3 lemniscate into a smooth surface:
real f(real x, real y, real z) {
  return lemn3(x,y)^2 + (16*abs((x,y))^4 + 1) * (z^2 - h^2);
}

draw(implicitsurface(f, (-3,-3,-3), (3,3,3), 
                     overlapedges=true),
     surfacepen=material(diffusepen=gray(0.5),
			                   emissivepen=gray(0.3),
			                   specularpen=gray(0.1)) );
\end{asypicture}
};
\end{tikzpicture}
\xdef\genusthreelisting{\asylistingfile}%
\end{minipage}
\par
\begin{minipage}[b]{\dimexpr\textwidth-9cm}
\subcaption{without setting \lstinline!overlapedges!}
\label{subfigure:nooverlapedges}
\end{minipage}
\begin{minipage}{8.8cm}
\begin{tikzpicture}
  [spy using outlines = {circle,
                         magnification=4,
                         size=2cm,
                         connect spies}]
\draw (0,0) node[above right]{%
\begin{asypicture}{name=genus_three_nooverlapedges}
settings.outformat = "png";
settings.render = 4;
size(8cm);

import smoothcontour3;

// Erdos lemniscate of order n:
real erdos(pair z, int n) { return abs(z^n-1)^2 - 1; }

real h = 0.12;

// Erdos lemniscate of order 3:
real lemn3(real x, real y) { return erdos((x,y), 3); }

// "Inflate" the order-3 lemniscate into a smooth surface:
real f(real x, real y, real z) {
  return lemn3(x,y)^2 + (16*abs((x,y))^4 + 1) * (z^2 - h^2);
}

draw(implicitsurface(f, (-3,-3,-3), (3,3,3)),
     surfacepen=material(diffusepen=gray(0.5),
			                   emissivepen=gray(0.3),
			                   specularpen=gray(0.1)) );
\end{asypicture}
};
\spy on (2.0,1.0) in node[below] at (5.0, 0.9);
\end{tikzpicture}
\end{minipage}
\caption{A smooth surface of genus three.}\label{figure:genus3}
\end{figure}
The first example is code for a smooth surface of genus three;
the output is Figure~\ref{figure:genus3}. Note that because of 
the adaptive nature of the algorithm, we can get a good result 
without having to set \lstinline!nx!, \lstinline!ny!, or 
\lstinline!nz! manually.
\lstinputlisting[firstline=5]{\genusthreelisting}
Also shown is the result obtained by omitting the parameter \lstinline!overlapedges=true!, with the rendering artifacts emphasized. The figure 
on top is not wholly devoid of such artifacts, but they are fewer and 
less prominent.
Increasing \lstinline!settings.render! can also reduce the appearance 
of such artifacts.

\subsection{Cropped sphere}
Next, we see how the module can be used to produce a cropping effect: If 
a surface can be described as the zero locus of a smooth function, then
the plotted zero locus is automatically cropped to the specified
rectangular solid.
\begin{asyexample}{3.5cm}
\begin{asypicture}{name=croppedsphere}
settings.outformat="png";
settings.render=8;
size(@mywidth,0);
import smoothcontour3;
currentprojection = orthographic(20,1,3);

real f(triple w) { return w.x^2 + w.y^2 + w.z^2 - 1; }

draw(implicitsurface(f, (-0.98,-0.98,-0.98), (0.98,0.98,0.98), 
             n=2, overlapedges=true),
     surfacepen=olive);
\end{asypicture}
\begincodelisting
\lstinputlisting[firstline=5,lastline=12]{\asylistingfile}
\breakcodelisting
\lstinputlisting[firstline=13]{\asylistingfile}
\end{asyexample}
\noindent The parameter \lstinline!n=2! is specified to produce a more efficient 
surface. This version of the cropped sphere has ``only'' 408 B\'ezier patches. 
This is in part because the algorithm decides that \lstinline!n=2! is actually 
to small and subdivides everything, so actually \lstinline!n=2! and 
\lstinline!n=4! give identical results. 
But by comparison, the default \lstinline!n=10!
gives 1152 patches, with no apparent change in the quality of the rendering.

Another thing to note here is that the algorithm would not ``notice'' the 
need to crop the sphere if the vertices were not positioned exactly right: 
in this case, \lstinline!n=2! produces better results than \lstinline!n=3!.
We do get good results for \lstinline!n=5!. If you change the cropping to 
$0.99$ instead of $0.98$, then \lstinline!n=2! still works great, but 
\lstinline!n=9! is the first \emph{odd} value of \lstinline!n! at which
the cropping shows up.

\subsection{Genus two surface}
\begin{figure}
\noindent
\begin{subfigure}{0.48\linewidth}
\centering
\begin{asypicture}{name=genus_two_with_gaps}
settings.outformat="png";
settings.render=8;
size(0.95*@the@linewidth, 0);
import smoothcontour3;
import contour;
currentprojection=perspective((18,20,10));

real tuberadius = 0.69;

// Convert to cylindrical coordinates to draw
// a circle revolved about the z axis.
real toruscontour(real x, real y, real z) {
  real r = sqrt(x^2 + y^2);
  return (r-2)^2 + z^2 - tuberadius^2;
}

// Take the union of the two tangent tori (by taking 
// the product of the functions defining them). Then
// add (or subtract) a bit of noise to smooth things 
// out.
real f(real x, real y, real z) {
  real f1 = toruscontour(x - 2 - tuberadius, y, z);
  real f2 = toruscontour(x + 2 + tuberadius, y, z);
  return f1 * f2 - 0.1;
}

// The noisy function extends a bit farther than the union of 
// the two tori, so include a bit of extra space in the box.
triple max = (2*(2+tuberadius), 2+tuberadius, tuberadius) + (0.1, 0.1, 0.1);

// Draw the implicit surface.
draw(implicitsurface(f, -max, max, overlapedges=true),
     surfacepen=white);
\end{asypicture}
\xdef\genustwowithgaps{\asylistingfile}%
\caption{default parameters}\label{subfigure:genus2withgaps}
\end{subfigure}
\hfill
\begin{subfigure}{0.48\linewidth}
\centering
\begin{asypicture}{name=genus_two}
settings.outformat="png";
settings.render=8;
size(0.95*@the@linewidth, 0);
import smoothcontour3;
import contour;
currentprojection=perspective((18,20,10));

real tuberadius = 0.69;

// Convert to cylindrical coordinates to draw
// a circle revolved about the z axis.
real toruscontour(real x, real y, real z) {
  real r = sqrt(x^2 + y^2);
  return (r-2)^2 + z^2 - tuberadius^2;
}

// Take the union of the two tangent tori (by taking 
// the product of the functions defining them). Then
// add (or subtract) a bit of noise to smooth things 
// out.
real f(real x, real y, real z) {
  real f1 = toruscontour(x - 2 - tuberadius, y, z);
  real f2 = toruscontour(x + 2 + tuberadius, y, z);
  return f1 * f2 - 0.1;
}

// The noisy function extends a bit farther than the union of 
// the two tori, so include a bit of extra space in the box.
triple max = (2*(2+tuberadius), 2+tuberadius, tuberadius) + (0.1, 0.1, 0.1);

// Draw the implicit surface.
draw(implicitsurface(f, -max, max, overlapedges=true, 
                     nx=20, nz=5),
     surfacepen=white);
\end{asypicture}
\xdef\genustwo{\asylistingfile}
\caption{\lstinline!nx=20!, \lstinline!nz=5!}\label{subfigure:genus2}
\end{subfigure}
\bigskip\par\noindent
\begin{minipage}{\textwidth}
\centering
\begin{asypicture}{name=genus_two_with_grid}
settings.outformat="png";
settings.render=8;
size(10cm,0);
import smoothcontour3;
import contour;
currentprojection=perspective((18,20,10));

real tuberadius = 0.69;

// Convert to cylindrical coordinates to draw
// a circle revolved about the z axis.
real toruscontour(real x, real y, real z) {
  real r = sqrt(x^2 + y^2);
  return (r-2)^2 + z^2 - tuberadius^2;
}

// Take the union of the two tangent tori (by taking 
// the product of the functions defining them). Then
// add (or subtract) a bit of noise to smooth things 
// out.
real f(real x, real y, real z) {
  real f1 = toruscontour(x - 2 - tuberadius, y, z);
  real f2 = toruscontour(x + 2 + tuberadius, y, z);
  return f1 * f2 - 0.1;
}

// The noisy function extends a bit farther than the union of 
// the two tori, so include a bit of extra space in the box.
triple max = (2*(2+tuberadius), 2+tuberadius, tuberadius) + (0.1, 0.1, 0.1);
triple min = -max;

// Draw the implicit surface.
draw(implicitsurface(f, min, max, overlapedges=true, 
                     nx=20, nz=5),
     surfacepen=material(green+0.8blue+0.1red,
			 emissivepen=0.2blue+0.2red,
			 specularpen=gray(0.1)));

/****************** WARNING **************************/
// The following code (for drawing gridlines) is highly 
// memory-intensive. On my computer, it used over 
// 1.7 GB of RAM.
/*****************************************************/

// To draw the gridlines, we need to compute the implicit curves
// in the planes defined by fixing one of the three variables.
typedef real function2(pair);
function2 xfixed(real x) { 
  return new real(pair p) { return f(x, p.x, p.y); }; 
}
function2 yfixed(real y) { 
  return new real(pair p) { return f(p.x, y, p.y); }; 
}
function2 zfixed(real z) { 
  return new real(pair p) { return f(p.x, p.y, z); }; 
}

// These planes are used to lift a path to a path3.
typedef triple plane(pair);
plane liftx(real x) { 
  return new triple(pair p) { return (x, p.x, p.y); }; 
}
plane lifty(real y) { 
  return new triple(pair p) { return (p.x, y, p.y); }; 
}
plane liftz(real z) { 
  return new triple(pair p) { return (p.x, p.y, z); }; 
}

pen gridpen = 0.8 red + 0.3 blue + linewidth(0.3pt);

int nx=20, ny=10, nz=5;

for (int i = 0; i <= nx; ++i) {
  real x = (i/nx) * (max.x - min.x) + min.x;
  guide[] wholepath = contour(xfixed(x), (min.y,min.z), 
                              (max.y,max.z),
                              new real[]{0}, 200)[0];
  plane currentplane = liftx(x);
  for (guide pathpiece : wholepath) {
    draw(path3(pathpiece, currentplane), gridpen);
  }
}

for (int j = 0; j <= ny; ++j) {
  real y = (j/ny) * (max.y - min.y) + min.y;
  guide[] wholepath = contour(yfixed(y), (min.x,min.z), 
                              (max.x,max.z), 
                              new real[]{0}, 200)[0];
  plane currentplane = lifty(y);
  for (guide pathpiece : wholepath) {
    draw(path3(pathpiece, currentplane), gridpen);
  }
}

for (int k = 0; k <= nz; ++k) {
  real z = (k/nz) * (max.z - min.z) + min.z;
  guide[] wholepath = contour(zfixed(z), (min.x,min.y), 
                              (max.x,max.y), 
                              new real[]{0}, 200)[0];
  plane currentplane = liftz(z);
  for (guide pathpiece : wholepath) {
    draw(path3(pathpiece, currentplane), gridpen);
  }
}
\end{asypicture}
\xdef\genustwowithgridlines{\asylistingfile}%
\subcaption{gridlines drawn using 
the \lstinline!contour! module}\label{subfigure:genus2withgridlines}
\end{minipage}
\caption{A surface of genus two.}
\end{figure}
Here's an attempt to draw a surface of genus two 
(with the result shown in Figure \ref{subfigure:genus2withgaps}):
\lstinputlisting[firstline=5]{\genustwowithgaps}
The algorithm, as you can see, is not always smart enough to identify 
all the things it should draw. In this case, the gaps can be eliminated 
by changing the $10 \times 10 \times 10$ grid used for initial computations 
to a $20 \times 10 \times 5$ grid. so that the individual cells are closer 
to being cubes. To do this, replace the final line by
\lstinputlisting[firstline=36]{\genustwo}
One more variant, shown in Figure \ref{subfigure:genus2withgridlines}, involves 
using the pre-existing \lstinline!contour! module to compute gridlines.
Note that this is highly memory-intensive.
\begin{lstlisting}[escapechar=*]

*\smash{\vdots}*
\end{lstlisting}
\vspace*{-\baselineskip}
\lstinputlisting[firstline=7,lastline=7]{\genustwowithgridlines}
\vspace*{-\baselineskip}
\begin{lstlisting}[escapechar=*]

*\smash{\vdots}*
\end{lstlisting}
\vspace*{-\baselineskip}
\lstinputlisting[firstline=33]{\genustwowithgridlines}

\section{Troubleshooting}
In most graphing algorithms in Asymptote, the user specifies the number of 
subdivisions, and many artifacts can be corrected by increasing this number.
For drawing smooth implicit surfaces, that approach proved impractical.
Instead, the algorithm  
attempts to detect artifacts itself, and subdivides into
smaller cells when it does.
This works surprisingly well, but not perfectly.

Most of the artifacts below can be mitigated by increasing 
the number of subdivisions using the parameter \lstinline!n!. 
However, simply increasing this parameter
should be regarded as a ``nuclear option.'' In addition to 
significantly increasing the computational resources required, 
it actually makes the first and most common artifact \emph{worse}.
%
\subsection{Random missing pixels}
Perhaps the most common artifact to show up when using this package 
is that scattered pixels will randomly show up the wrong color; see
Figure \ref{subfigure:nooverlapedges}, p.~\pageref{subfigure:nooverlapedges}.
These pixels are actually showing up at the edges between patches, where the renderer 
believes (usually mistakenly) that there is a gap it can ``see through.'' There are 
several ways to mitigate the problem; they can be used together except for the last.
\begin{enumerate}
\item Set \lstinline!overlapedges = true! in the call to 
\lstinline!implicitsurface()!. This 
option artificially inflates the individual patches by $1\%$ in an effort to convince the 
renderer that all the gaps between edges are filled. It usually helps a lot.
\item Increase \lstinline!settings.render!. As a general rule, asking the renderer to work 
harder will mitigate the random missing pixels.
\item Decrease the parameters 
\lstinline!n! and/or \lstinline!nx!, \lstinline!ny!, \lstinline!nz!. If you can decrease the number 
of patches (which is not guaranteed), then you correspondingly get fewer edges to produce problems.
Also, if the patches are larger (since fewer), then \lstinline!overlapedges=true! becomes more effective
since a $1\%$ increase in the size of the patches produces a bigger increase in absolute terms.
\item When drawing the surface, use a \lstinline!meshpen!. This is not recommended---the mesh you 
get will not be one you want to see---but it will fill the gaps. Do \emph{not} use this with
\lstinline!overlapedges=true!.
\end{enumerate}
%
\subsection{Gaps in the surface}\label{subsection:gapsinsurface}
Sometimes, the algorithm will fail to notice a piece of the surface that should 
be drawn, leaving a gap as in Figure \ref{subfigure:genus2withgaps}.
\begin{enumerate}
\item \textbf{Jiggle it a little.} Change small things to see if you can remove 
the coincidence that stumped the algorithm. For instance, add or subtract a 
small amount from the bounds, or change \lstinline!n! by one. In 
the case of Figure \ref{subfigure:genus2withgaps}, subtracting $0.05$ from the 
upper $z$ bound (and adding $0.05$ to the lower $z$ bound) is sufficient to 
eliminate the problem.
\item \textbf{Judiciously change \lstinline!nx!, \lstinline!ny!, \lstinline!nz!.}
For instance, if $z$ can be described as a function of $(x,y)$, try setting 
\lstinline!nz=1!.
In the case of Figure \ref{subfigure:genus2withgaps}, the gaps are long 
in the $x$ direction, so you might try doubling \lstinline!nx! to $20$ from 
the default of $10$. Since the gaps have little or no extent in the 
$z$ direction, and since the figure itself is not high in the $z$ direction, 
you might simultaneously decrease \lstinline!nz! from $10$ to $5$. In this 
way, you have solved the issue (producing 
Figure~\ref{subfigure:genus2}). Furthermore, since 
$20 \cdot 10 \cdot 5 = 10 \cdot 10 \cdot 10$, you have not actually 
increased the computation time.
\item \textbf{Significantly increase \lstinline!n!.} Theoretically, if 
the surface is smooth, \lstinline!n! is sufficiently large, and a small random 
amount of random noise is added to the bounds, errors of this sort should occur 
only with probability zero.\footnote{This is true for a suitably high level of 
abstraction. At a lower level of abstraction, a computer is a finite-state 
machine, so there's no such thing as ``possible but with probability zero.''}
\end{enumerate}
\subsection{Important features drawn wrong}
For instance, a ``cropped sphere'' might fail to be cropped, or a very tiny 
sphere might go missing from the diagram. Or bad things could happen if the 
surface is supposed to have multiple ``sheets'' in close proximity. 

The advice here is the same as in \ref{subsection:gapsinsurface}; 
however, this time the problem
is too vaguely worded for me to offer any ``probability zero'' guarantees.
\subsection{Strange smoothness artifacts}
\begin{figure}
\centering
\begin{asypicture}{name=conewithspyglass}
settings.outformat="pdf";
settings.render=16;
settings.prc=false;
size(10cm);

import smoothcontour3;

currentprojection=perspective(
camera=(2.02192872015484,4.99392113582712,3.92208163966281),
up=(-0.000455904165055687,-0.001742164203227,0.00236915601403234),
target=(0.028573554295769,0.0228009523917252,-0.117030854731037));

real f(real x, real y, real z) {
  return x^2 + y^2 - z^2;
}

draw(implicitsurface(f, (-1,-1,-1), (1,1,1), overlapedges=true),
     surfacepen=material(gray(0.5)));

//dot(0.4*(1,0,-1));

frame theframe = currentpicture.fit();
currentpicture = new picture;
unitsize(1cm);

pair thepoint = (0.2,-1.8);
path thecircle = circle(c=thepoint, r=0.3);
path bigcircle = scale(4) * thecircle;
add(scale(4) * theframe);
layer();
clip(bigcircle);
transform t = shift(5,0) * shift(-3 thepoint);
currentpicture = t * currentpicture;
bigcircle = t * bigcircle;
pair center2 = t * scale(4) * thepoint;
add(theframe);
layer();
draw(bigcircle, blue);
draw(thecircle, blue);
draw(lastcut(firstcut(thepoint -- center2, thecircle).after, bigcircle).before,
     blue);
\end{asypicture}
\caption{The graph of $x^2 + y^2 - z^2 = 0$ over the domain 
$[-1,1]^3$. Note the minor wrinkle.}\label{figure:conewithspyglass}
\end{figure}
Sometimes strange artifacts show up, such as the wrinkle in 
Figure~\ref{figure:conewithspyglass}. To deal with this, you can try the same 
\ref{subsection:gapsinsurface}, but with absolutely no assurance of 
success. Another thing you can try is to play with the coloring, lighting, 
and point of view to make the artifact less obvious. (In truth, it required 
some playing with coloring, lighting, and point of view to make the artifact 
visible in Figure~\ref{figure:conewithspyglass}.)

One particular piece of advice here: a strong \lstinline!specularpen! can 
highlight any such artifacts if they actually exist. So if you are trying to 
de-emphasize an artifact, decreasing the \lstinline!specularpen! and increasing 
the other pens in the \lstinline!material! may be wise.

\section{License}
The following license applies to the both the \lstinline!smoothcontour3! module and this 
documentation:
\begin{verbatim}
Copyright 2015 Charles Staats III

Licensed under the Apache License, Version 2.0 (the "License");
you may not use this file except in compliance with the License.
You may obtain a copy of the License at

    http://www.apache.org/licenses/LICENSE-2.0

Unless required by applicable law or agreed to in writing, software
distributed under the License is distributed on an "AS IS" BASIS,
WITHOUT WARRANTIES OR CONDITIONS OF ANY KIND, either express or implied.
See the License for the specific language governing permissions and
limitations under the License.
\end{verbatim}
\end{document}